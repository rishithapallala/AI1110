\documentclass[12pt,onecolumn,notitlepage]{article}
\usepackage[margin=0.5in]{geometry}
\usepackage{amsmath}
\usepackage{gensymb}
\usepackage{graphicx}
\usepackage{amsthm}
\usepackage{mathrsfs}
\usepackage{txfonts}
\usepackage{cite}
\usepackage{cases}
\usepackage{subfig}
\usepackage[breaklinks=true]{hyperref}
\usepackage{listings}
\usepackage[latin1]{inputenc}
\usepackage{color}
\usepackage{array}
\usepackage{longtable}
\usepackage{calc}
\usepackage{multirow}
\usepackage{hhline}
\usepackage{ifthen}
\usepackage{amssymb}
\usepackage{multicol}

\providecommand{\pr}[1]{\ensuremath{\Pr\left(#1\right)}}
\providecommand{\sbrak}[1]{\ensuremath{{}\left[#1\right]}}
\providecommand{\lsbrak}[1]{\ensuremath{{}\left[#1\right.}}
\providecommand{\rsbrak}[1]{\ensuremath{{}\left.#1\right]}}
\providecommand{\brak}[1]{\ensuremath{\left(#1\right)}}
\providecommand{\lbrak}[1]{\ensuremath{\left(#1\right.}}
\providecommand{\rbrak}[1]{\ensuremath{\left.#1\right)}}
\providecommand{\cbrak}[1]{\ensuremath{\left\{#1\right\}}}
\providecommand{\lcbrak}[1]{\ensuremath{\left\{#1\right.}}
\providecommand{\rcbrak}[1]{\ensuremath{\left.#1\right\}}}
\newcommand*{\comb}[2]{{}^{#1}C_{#2}}
\title{Probability Assignment 2 (11.16.3.7)}
\author{Pallala Rishitha (BT22BTECH11011)}
\date{}
\begin{document}
\maketitle
\textbf{Question}
A fair coin is tossed four times , and a person win Re 1 for each head and lose Rs 1.50 for each tail that turns up.From the sample space calculate how many different amounts of money you can have after four tosses and the
probability of having each of these amounts. \\
\textbf{Solution}
Let X denote the number of heads obtained after the 4 tosses. Clearly, X has the binomial distribution with $n=4$ and p being the probability of obtaining a head.
\begin{align}
    p &= \frac{1}{2} \\ 
    q &=1-p = \frac{1}{2}     
\end{align}
Now, since X has the binomial distribution,
\begin{align}
 Pr\brak{X=r} &= \comb{n}{r}\brak{p}^{r}\brak{q}^{n-r} 
\end{align}
Let Y be the amount obtained after 4 tosses
\begin{align}
    Y= (1\times X) - (1.5\times(4-X))
\end{align}
As Y = Q\brak{X},
\begin{align}
  Pr\brak{Y=Y_0} = \sum_i Pr\brak{X=i} &\brak{\forall i\in [0,4]:Q\brak{i}=Y_0} 
 \end{align}
 The Table \ref{table:1} shows parameters in the solution along with their definition and Values.\\
 \setlength{\tabcolsep}{16pt}
 \renewcommand{\arraystretch}{2.15}
 \begin{table}[h!]
\centering
\caption{PARAMETER DECLARATION}
\label{table:1}

\begin{tabular}{|c|c|c|}
\hline
Parameters  & Description  & Values\\
\hline
n&Number of trials&4                           \\
\hline
p&probability of sucessful trial&$ \dfrac{1}{2}$\\
\hline
q&probability of unsucessful trial&$ \dfrac{1}{2}$\\
\hline
X&Random variable for the number of heads&0,1,2,3,4\\
\hline
Y&Random variable for amount obtained after 4 trials&-6,-3.5,-1,1.5,4\\ 
\hline



\end{tabular}

\end{table}

Now, since X has the binomial distribution, the Probability mass function($Pr\brak{X=r}$) and  cummulative distribution function($F_X\brak{r}$) is given by
\begin{align}
 Pr\brak{X=r} &= \comb{n}{r}\brak{p}^{r}\brak{q}^{n-r} 
\end{align}
\begin{align}
    F_X\brak{r} &= Pr\brak{X \le r}
\end{align}

\begin{align}
    \therefore   F_X(r)=\sum_{i=0}^r\comb{n}{i}p^iq^{n-i}
\end{align}
 \setlength{\tabcolsep}{18pt}
 \renewcommand{\arraystretch}{2.15}
\begin{table}[h!]
\centering

\begin{tabular}{|c|c|c|}
\hline
S.no	&Y	&Pr(Y)\\
\hline
1	&-6	&$ \dfrac{1}{16}$\\\hline
2	&-3.5	&$ \dfrac{4}{16}$\\\hline
3	&-1	&$\dfrac{6}{16}$\\\hline
4	&1.5	&$\dfrac{4}{16}$\\\hline
5	&4	&$ \dfrac{1}{16}$\\
\hline
\end{tabular}

\caption{PMF and CDF of Y}
\label{table:2}
\end{table}
The Table \ref{table:2} shows the probability of different amounts of money after four tosses(PMF) and CDF.\\

\end{document}
