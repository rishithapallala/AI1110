\documentclass[12pt,twocolumn,notitlepage]{article}
\usepackage[margin=0.5in]{geometry}
\usepackage{amsmath}
\usepackage{gensymb}
\usepackage{graphicx}
\usepackage{amsthm}
\usepackage{mathrsfs}
\usepackage{txfonts}
\usepackage{cite}
\usepackage{cases}
\usepackage{subfig}
\usepackage[breaklinks=true]{hyperref}
\usepackage{listings}
\usepackage[latin1]{inputenc}
\usepackage{color}
\usepackage{array}
\usepackage{longtable}
\usepackage{calc}
\usepackage{multirow}
\usepackage{hhline}
\usepackage{ifthen}
\usepackage{amssymb}
\providecommand{\pr}[1]{\ensuremath{\Pr\left(#1\right)}}
\providecommand{\sbrak}[1]{\ensuremath{{}\left[#1\right]}}
\providecommand{\lsbrak}[1]{\ensuremath{{}\left[#1\right.}}
\providecommand{\rsbrak}[1]{\ensuremath{{}\left.#1\right]}}
\providecommand{\brak}[1]{\ensuremath{\left(#1\right)}}
\providecommand{\lbrak}[1]{\ensuremath{\left(#1\right.}}
\providecommand{\rbrak}[1]{\ensuremath{\left.#1\right)}}
\providecommand{\cbrak}[1]{\ensuremath{\left\{#1\right\}}}
\providecommand{\lcbrak}[1]{\ensuremath{\left\{#1\right.}}
\providecommand{\rcbrak}[1]{\ensuremath{\left.#1\right\}}}

\newcommand*{\comb}[2]{{}^{#1}C_{#2}}

\title{Probability Assignment 1 (12.13.5.12)}
\author{Pallala Rishitha (BT22BTECH11011)}
\date{}

\begin{document}

\maketitle
\subsection*{Question}
Find the probability of throwing at most 2 sixes in 6 throws of a single die.\\



\subsection*{Solution}

Let X denote the number of sixes obtained after the 6 trials. Clearly, X has the binomial distribution with $n=6$ and p being the probability of obtaining a six ,
\begin{align}
    p &= \frac{1}{6}      
\end{align}

Now, since X has the binomial distribution, the probability mass function is given by
\begin{align}
    P_X\brak{r} &= \comb{n}{r}\brak{\frac{1}{6}}^{r}\brak{\frac{5}{6}}^{n-r} 
\end{align}
Substituting the values of r as 0,1,2 :
\begin{align}
    P_X\brak{0} &= \comb{6}{0}\brak{\frac{1}{6}}^{0}\brak{\frac{5}{6}}^{6-0} 
\\
   &=\frac{15625}{46656}
\end{align}
\begin{align}
    P_X\brak{1} &= \comb{6}{1}\brak{\frac{1}{6}}^{1}\brak{\frac{5}{6}}^{6-1} 
\\
   &=\frac{18750}{46656}
\end{align}
\begin{align}
    P_X\brak{2} &= \comb{6}{2}\brak{\frac{1}{6}}^{2}\brak{\frac{5}{6}}^{6-2} 
\\
   &=\frac{9375}{46656}
\end{align}


Hence, the probability of throwing at most 2 sixes is
\begin{align}
    P_X\brak{<=2} &= P_X\brak{0} + P_X\brak{1} + P_X\brak{2}  \\
    &= \frac{21875}{23328} 
\end{align}

\end{document}
